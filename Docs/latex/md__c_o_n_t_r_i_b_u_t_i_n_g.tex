\+:+1\+::tada\+: First off, thanks for taking the time to contribute! \+:tada\+:\+:+1\+:

The following is a set of guidelines for contributing to railway-\/dot-\/exe.

\paragraph*{Table Of Contents}

\href{#code-of-conduct}{\tt Code of Conduct}

\href{#i-dont-want-to-read-this-whole-thing-i-just-have-a-question}{\tt I don\textquotesingle{}t want to read this whole thing, I just have a question!!!}

\href{#what-should-i-know-before-i-get-started-as-a-contributor}{\tt What should I know before I get started as a contributor?}
\begin{DoxyItemize}
\item \href{#railway-dot-exe-and-c++}{\tt Railway-\/dot-\/exe and C++}
\end{DoxyItemize}

\href{#how-can-i-contribute}{\tt How Can I Contribute?}
\begin{DoxyItemize}
\item \href{#reporting-bugs}{\tt Reporting Bugs}
\item \href{#pull-requests}{\tt Pull Requests}
\item \href{#suggested-enhancements}{\tt Suggested Enhancements}
\end{DoxyItemize}

\href{#styleguides}{\tt Styleguides}
\begin{DoxyItemize}
\item \href{#git-commit-messages}{\tt Git Commit Messages}
\item \href{#editing-the-developer-guide}{\tt Editing the Developer Guide}
\end{DoxyItemize}

\href{#attributions}{\tt Attributions}

\subsection*{Code of Conduct}

Be nice to each other. Please report unacceptable behaviour to \href{mailto:railwayfeedback@gmail.com}{\tt railwayfeedback@gmail.\+com}.

\subsection*{I don\textquotesingle{}t want to read this whole thing I just have a question!!!}

We have an official \href{http://www.railwayoperationsimulator.com}{\tt website} where you can ask questions via the \href{http://www.railwayoperationsimulator.com/contact}{\tt Contact} form, or simply \href{mailto:railwayfeedback@gmail.com}{\tt Email us}.

\subsection*{What should I know before I get started as a contributor?}

\subsubsection*{Railway-\/dot-\/exe and C++}

Railway-\/dot-\/exe is a large open source project --- and it\textquotesingle{}s written in C++. In order to make changes you will need to have some competence in C++ programming. If you want to experiment or learn C++ using railway-\/dot-\/exe by all means please do --- but fork your own copy of the project. Once you feel confident with your level of experience you can submit a pull-\/request via Git\+Hub. Many people use the software, and have for many years, so you don\textquotesingle{}t want to break things on the master repository because you\textquotesingle{}re learning or trying new things. Once we see evidence of proficiency we\textquotesingle{}ll give you write access to the repository.

The program is written using Embarcardero\textquotesingle{}s C++ Builder Starter Edition. As such, it\textquotesingle{}s only able to produce a win32 version of the program. This will happily run on your 64bit Windows machine though. Plans are underway (as of February 2018) to move to a new C++ platform that will support other operating systems (Linux/\+Mac\+O\+S/etc.) without the associated cost of upgrading your C++ Builder edition.

Until then, simply visit Embarcadero\textquotesingle{}s site to download the free C++ Builder Starter Edition. Then {\ttfamily git clone} our repository and open the railway.\+cbproj file in C++ Builder.


\begin{DoxyItemize}
\item \href{https://www.embarcadero.com/products/cbuilder/start-for-free}{\tt Embarcadero C++ Starter Edition}
\end{DoxyItemize}

\subsection*{How Can I Contribute?}

\subsubsection*{Reporting Bugs}

Use the Issues tab on our project page to report a bug. See the guide \href{https://guides.github.com/features/issues/}{\tt Git\+Hub issues} if you need help with creating an Issue .

\paragraph*{How Do I Submit A (Good) Bug Report?}

Explain the problem and include additional details to help maintainers reproduce the problem\+:


\begin{DoxyItemize}
\item {\bfseries Use a clear and descriptive title} for the issue to identify the problem.
\item {\bfseries Describe the exact steps which reproduce the problem} in as much detail as possible.
\item {\bfseries Explain which behavior you expected to see instead and why.}
\item {\bfseries Include screenshots if possible}.
\end{DoxyItemize}

Include details about your environment\+:


\begin{DoxyItemize}
\item {\bfseries Which version of railway-\/dot-\/exe are you using?} You can get the exact version by opening the About form using the Help menu.
\item {\bfseries What\textquotesingle{}s the name and version of the OS you are using}?
\item {\bfseries Are you running railway-\/dot-\/exe in a virtual machine?} If so, which VM software are you using and which operating systems and versions are used for the host and the guest?
\end{DoxyItemize}

Optionally, provide more context by answering these questions\+:


\begin{DoxyItemize}
\item {\bfseries Did the problem start happening recently} (e.\+g. after updating to a new version of railway-\/dot-\/exe) or was this always a problem?
\item If the problem started happening recently, {\bfseries can you reproduce the problem in an older version of railway-\/dot-\/exe?} What\textquotesingle{}s the most recent version in which the problem doesn\textquotesingle{}t happen? You can download older versions of railway-\/dot-\/exe from the official site\textquotesingle{}s download page link to \href{https://www.dropbox.com/sh/wvruss55cfzdvgw/AAApyZeGaIRyJAtS6clOuo0La?dl=0}{\tt earlier versions}.
\item {\bfseries Can you reliably reproduce the issue?} If not, provide details about how often the problem happens and under which conditions it normally happens.
\end{DoxyItemize}

\subsubsection*{Pull Requests}


\begin{DoxyItemize}
\item There is no template for pull requests at this time. Simply create your pull request and try to include a meaningful description of what you are proposing.
\end{DoxyItemize}

\subsubsection*{Suggested Enhancements}

Although the program is fully usable now, there remains great potential for further development. Some ideas include\+:


\begin{DoxyItemize}
\item automatic route setting
\item option to use \& display imperial units as well as metric units
\item ability to record and replay sessions
\item addition of automatic signal routes over level crossings, with train-\/triggered crossing opening
\item multi-\/player operation over the internet
\item signalbox mode where individual signals and points are operated directly, perhaps via a graphical lever frame, with user-\/defined interlocking
\item sound effects, e.\+g. enter a track ID \& hear trains \& station announcements at that location
\item random failures of trains, signals \& points etc, and random delay times at stations
\item incorporation of user-\/defined graphics
\item variable train lengths
\item restricted routes – e.\+g. DC 3 \& 4 rail, 25kV AC, tube lines and so on
\item etc. etc.
\end{DoxyItemize}

So… there’s plenty still to do… what are you waiting for?

\subsection*{Styleguides}

\subsubsection*{Git Commit Messages}


\begin{DoxyItemize}
\item Limit the first line to 72 characters or less
\item Use the present tense (\char`\"{}\+Add feature\char`\"{} not \char`\"{}\+Added feature\char`\"{})
\item Use the imperative mood (\char`\"{}\+Move icon to...\char`\"{} not \char`\"{}\+Moves icon to...\char`\"{})
\item A properly formatted commit subject line should always be able to complete the following sentence\+:
\begin{DoxyItemize}
\item This commit will {\bfseries your commit message here}
\end{DoxyItemize}
\end{DoxyItemize}

\subsubsection*{Editing the Developer Guide}


\begin{DoxyItemize}
\item If you want access to the source copy of the Developer Guide (used to produce the P\+DF version on Git\+Hub) just \href{mailto:railwayfeedback@gmail.com}{\tt Email us} asking for access.
\end{DoxyItemize}

\subsection*{Attributions}

Menu icons from Silk Icon Set 1.\+3 by Mark James used under Creative Commons Attribution 2.\+5 License

\href{http://creativecommons.org/licenses/by/2.5/}{\tt http\+://creativecommons.\+org/licenses/by/2.\+5/} 