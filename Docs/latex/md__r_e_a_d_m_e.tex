Welcome to the railway.\+exe project, a railway simulator for Windows originally written in Borland\textquotesingle{}s C++ Builder 4 Professional and now updated to work with Embarcadero\textquotesingle{}s C++ Builder Starter Edition v10.\+2. Development to date has resulted in a complete and usable program that provides options to\+:


\begin{DoxyItemize}
\item build a railway of any size
\item add text of any available colour, font and size
\item set preferred and bi-\/directional running directions
\item choose light or dark backgrounds
\item develop timetables with shuttle services, changes in direction, splits, joins, and repeating services
\item operate trains
\item zoom-\/out for a wider display
\end{DoxyItemize}

A wide selection of track element types is available for building the railway, together with station elements consisting of platforms, concourses and footbridges. Also available are non-\/station named location elements for sidings, works, depots, junction approaches and anything else that needs a name. Location names are displayed on the railway in user-\/selectable font, style, size and colour. In addition to named locations any other text may be added to the railway, again in any font, style, size and colour. Location names and other text may be moved in order to improve the appearance of the railway and areas of track may be selected and cut, copied, pasted, deleted, mirrored and flipped etc. Railway files may be saved and loaded in both development form during construction and in operational form on completion. Track element lengths and line speed limits may be set individually, along tracks, or in areas by selection, as can preferred running directions.

Trains may operate to a timetable developed using the internal timetable editor, or under signaller control. Three types of route are available\+:


\begin{DoxyEnumerate}
\item Automatic signal routes\+: set signal to signal in preferred directions, the route is retained after trains pass and signals automatically return to green in stages as blocks ahead are cleared
\item Preferred direction routes\+: set signal to signal in preferred directions, and the route is cleared as trains pass
\item Unrestricted routes\+: set from most types of track element to other track elements in any direction. The route is cleared as trains pass
\end{DoxyEnumerate}

In addition trains will run on track that has no route set, but they are then much more vulnerable to derailments and crashes.

Although the program is fully usable now, there remains great potential for further development. Some ideas include\+:


\begin{DoxyItemize}
\item automatic route setting
\item option to use \& display imperial units as well as metric units
\item ability to record and replay sessions
\item signalbox mode where individual signals and points are operated directly, perhaps via a graphical lever frame, with user-\/defined interlocking
\item sound effects, e.\+g. enter a track ID \& hear trains \& station announcements at that location
\item random failures of trains, signals \& points, etc.
\item incorporation of user-\/defined graphics
\item variable train lengths
\item restricted routes -\/ e.\+g. DC 3 \& 4 rail, 25kV AC, tube lines, etc.
\item multi-\/player operation over the internet
\end{DoxyItemize}

More recent suggestions are included in the file Dev\+History.\+txt.

\textquotesingle{}Developer\+Guide.\+pdf\textquotesingle{} in the \textquotesingle{}master\textquotesingle{} branch provides a summary of program history, structure, operation and interfaces, and also explains the file structure. Anyone wishing to contribute to further development should create a Git\+Hub account then contact me with evidence of proficiency and I\textquotesingle{}ll provide write access to the files. Version control is provided by \textquotesingle{}Git\textquotesingle{}, so anyone wishing to contribute needs to be familiar with \textquotesingle{}Git\textquotesingle{} -\/ plenty of information is available on the web and it\textquotesingle{}s not difficult to learn the basics.

I (Albert Ball) can be contacted at \href{mailto:railwayfeedback@gmail.com}{\tt railwayfeedback@gmail.\+com} or via the website at \href{http://www.railwayoperationsimulator.com/}{\tt http\+://www.\+railwayoperationsimulator.\+com/} 